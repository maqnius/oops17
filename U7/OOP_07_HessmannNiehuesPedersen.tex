\input{../src/header}											% bindet Header ein (WICHTIG)

\newcommand{\dozent}{Prof. Dr. Claudia Müller-Birn, Barry Linnert}					% <-- Names des Dozenten eintragen
\newcommand{\tutor}{Thierry Meurers}						% <-- Name eurer Tutoriun eintragen
\newcommand{\tutoriumNo}{10}				% <-- Nummer im KVV nachschauen
\newcommand{\ubungNo}{06}									% <-- Nummer des Übungszettels
\newcommand{\veranstaltung}{Objektorientierte Programmierung}	% <-- Name der Lehrveranstaltung eintragen
\newcommand{\semester}{SoSe 17}						% <-- z.B. SoSo 17, WiSe 17/18
\newcommand{\studenten}{Stefaan Hessmann, Jaap Pedersen, Mark Niehues}			% <-- Hier eure Namen eintragen

% /////////////////////// BEGIN DOKUMENT /////////////////////////
\begin{document}
\input{../src/titlepage}										% erstellt die Titelseite


\section{Aufgabe 1}

\begin{description}
\item[a)] Die von Python bereitgestellte sorted()-Funktion nutzt einen hybriden Sortier-Algorithmus (Mix aus Mergesort und Insertionsort) namens Timsort.
\item[d)]
Im Vergleich zu den anderen vergleichsbasierten Sortieralgorithmen, die in der Vorlesung besprochen wurden, erzielt Timsort immer die bestmögliche Laufzeit für best-, average- und worst-case. In der Tabelle befindet sich kein anderer Algorithmus der in den drei Fällen die beste Laufzeit liefert. Timsort ist außerdem ein stabiler Sortieralgorithmus, sodass Elemente mit gleichem Schlüssel nicht vertauscht werden. \\
Aufgrund seiner Laufzeit und Stabilität eignet sich Timsort gut als Standard-Sortieralgorithmus für die Python Umgebung, da die Sortierung maximal schnell abläuft und zusätzlich die Sortierung von gleichen Schlüsseln beibehalten wird. Diese Eigenschaft ist besonders beim Sortieren nach mehreren Schlüsseln (beispielsweise Tabellen) notwendig.
\begin{table}[H]
\centering
\begin{tabular}{|c|c|c|c|c|c|}
\hline
Algorithmus & best & average & worst & in-place & stabil \\ \hline
Selection Sort & $n^2$ & $n^2$ & $n^2$ & Ja & Nein \\ \hline
Insertion Sort & $n$ &$ n^2$ &$ n^2$ & Ja & Ja \\ \hline
Bubble Sort & $n$ & $n^2$ & $n^2$ & Ja & Ja \\ \hline
Quick Sort & $n log(n)$ & $n log(n)$ & $n^2$ & Nein & Nein \\ \hline
Merge Sort & $n log(n)$ & $n log(n)$ & $n log(n)$ & Nein & Ja \\ \hline
Heap Sort & $n log(n)$ & $n log(n)$ & $n log(n)$ & Ja & Nein \\ \hline
Timsort & $n$ & $n log(n)$ & $n log(n)$ & Nein & Ja \\ \hline
\end{tabular}
\caption{Vergleichsbasierte Sortieralgorithmen}
\end{table}
\end{description}





%/////////////////////// END DOKUMENT %/////////////////////////
\end{document}
