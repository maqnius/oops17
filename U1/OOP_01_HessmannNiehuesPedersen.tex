\input{../src/header}											% bindet Header ein (WICHTIG)

\newcommand{\dozent}{Prof. Dr. Claudia Müller-Birn, Barry Linnert}					% <-- Names des Dozenten eintragen
\newcommand{\tutor}{Thierry Meurers}						% <-- Name eurer Tutoriun eintragen
\newcommand{\tutoriumNo}{10}				% <-- Nummer im KVV nachschauen
\newcommand{\ubungNo}{01}									% <-- Nummer des Übungszettels
\newcommand{\veranstaltung}{Objektorientierte Programmierung}	% <-- Name der Lehrveranstaltung eintragen
\newcommand{\semester}{SoSe 17}						% <-- z.B. SoSo 17, WiSe 17/18
\newcommand{\studenten}{Stefaan Hessmann, Jaap Pedersen, Mark Niehues}			% <-- Hier eure Namen eintragen

% /////////////////////// BEGIN DOKUMENT /////////////////////////
\begin{document}
\input{../src/titlepage}										% erstellt die Titelseite


% /////////////////////// Aufgabe 1 /////////////////////////
\section{Aufgabe: Wochentage \hfill (10 Punkte)}
{\itshape Berechnet den Wochentag anhand des Datums}

\lstinputlisting[style=py,									% Style
	caption={Beispiel: Code zu Aufgabe 1},		% Beschriftung
	firstnumber={1},										% Start der Nummerierung
	firstline={1}											% 1. Codezeile
]											% letzte Codezeile
{./src/aufgabe_1.py}


% ///////////////////// Aufgabe 2 ///////////////////
\section{Aufgabe: Summen berechnen \hfill (12 Punkte)}
\lstinputlisting[style=py,									% Style
	caption={Beispiel: Code zu Aufgabe 2},		% Beschriftung
	firstnumber={1},										% Start der Nummerierung
	firstline={12}											% 1. Codezeile
]											% letzte Codezeile
{./src/aufgabe_2.py}

\section{Aufgabe: Multiplikation \hfill (8 Punkte)}

\lstinputlisting[style=py,									% Style
	caption={Beispiel: Code zu Aufgabe 3},		% Beschriftung
	firstnumber={1},										% Start der Nummerierung
	firstline={1}											% 1. Codezeile
]											% letzte Codezeile
{./src/aufgabe_3.py}
% /////////////////////// END DOKUMENT /////////////////////////
\end{document}
