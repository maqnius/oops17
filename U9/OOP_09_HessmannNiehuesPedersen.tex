\input{../src/header}											% bindet Header ein (WICHTIG)

\newcommand{\dozent}{Prof. Dr. Claudia Müller-Birn, Barry Linnert}					% <-- Names des Dozenten eintragen
\newcommand{\tutor}{Thierry Meurers}						% <-- Name eurer Tutoriun eintragen
\newcommand{\tutoriumNo}{10}				% <-- Nummer im KVV nachschauen
\newcommand{\ubungNo}{09}									% <-- Nummer des Übungszettels
\newcommand{\veranstaltung}{Objektorientierte Programmierung}	% <-- Name der Lehrveranstaltung eintragen
\newcommand{\semester}{SoSe 17}						% <-- z.B. SoSo 17, WiSe 17/18
\newcommand{\studenten}{Stefaan Hessmann, Jaap Pedersen, Mark Niehues}			% <-- Hier eure Namen eintragen

% /////////////////////// BEGIN DOKUMENT /////////////////////////
\begin{document}
\input{../src/titlepage}										% erstellt die Titelseite


\section{Aufgabe 1}
\lstinputlisting[style=java,									% Style
	caption={House Class},		% Beschriftung
	firstnumber={0},										% Start der Nummerierung
	firstline={0}											% 1. Codezeile
]											% letzte Codezeile
{./src/u9/src/House.java}


\lstinputlisting[style=java,									% Style
	caption={The Street Class organizes the houses},		% Beschriftung
	firstnumber={0},										% Start der Nummerierung
	firstline={0}											% 1. Codezeile
]											% letzte Codezeile
{./src/u9/src/Street.java}

\lstinputlisting[style=java,									% Style
	caption={The Main routine that runs an example simulation with output},		% Beschriftung
	firstnumber={0},										% Start der Nummerierung
	firstline={0}											% 1. Codezeile
]											% letzte Codezeile
{./src/u9/src/Main.java}

\section{Aufgabe 2}

\subsection{Beobachtungen}
\begin{itemize}
\item a) ist immer $1.0$, da der erste Summand $1.0$ ist und die darauf folgenden auf 0 gerundet werden durch die division zweier integers.
\item die Summe für b) ist erst $1.0$ und ab $n = 10^6$ erfolgt ein Abbruch, da der Wertebereich für Integers in Java zyklisch implementiert ist, sodass irgendwann $i*i=0$ ist und die Division durch $0$ für Integer nicht definiert ist.
\item Im Gegensatz dazu ist die Division durch $0$ für double definiert als \texttt{infinity}, sodass ab $n = 10^6$ im Falle d) und f) das Ergebnis \texttt{infinity} ist.
\item Die anderen Varianten funktionieren, aber weisen zum Teil leicht unterschiedliche Abweichungen auf.
\end{itemize}

\subsection{Ergebnisse der verschiedenen Durchläufe}
\begin{table}[ht] 
 \begin{tabular}{l | l | l} 
 
Variante & Endwert & Abweichung vom Grenzwert \\ 
 \hline 

a) &1.0&0.6449340668482264\\ 
b) &1.0&0.6449340668482264\\ 
c) &1.6349839001848923&0.009950166663334148\\ 
d) &1.6349839001848923&0.009950166663334148\\ 
e) &1.6349839001848923&0.009950166663334148\\ 
f) &1.6349839001848923&0.009950166663334148\\ 
g) &1.6349839032409363&0.009950163607290063\\ 
h) &1.6349839001848923&0.009950166663334148\\ 
\end{tabular} 
 \caption{n = 100} 
 \end{table}



\begin{table}[ht] 
 \begin{tabular}{l | l | l} 
 
Variante & Endwert & Abweichung vom Grenzwert \\ 
 \hline 

a) &1.0&0.6449340668482264\\ 
b) &1.0&0.6449340668482264\\ 
c) &1.6448340718480652&9.999500016122376E-5\\ 
d) &1.6448340718480652&9.999500016122376E-5\\ 
e) &1.6448340718480652&9.999500016122376E-5\\ 
f) &1.6448340718480652&9.999500016122376E-5\\ 
g) &1.644834074928367&9.999191985943234E-5\\ 
h) &1.6448340718480652&9.999500016122376E-5\\ 
\end{tabular} 
 \caption{n = 10000} 
 \end{table}



\begin{table}[ht] 
 \begin{tabular}{l | l | l} 
 
Variante & Endwert & Abweichung vom Grenzwert \\ 
 \hline 

a) &1.0&0.6449340668482264\\ 
b) &Abgebrochen&Abgebrochen\\ 
c) &1.64493306684877&9.999994563525405E-7\\ 
d) &Infinity&-Infinity\\ 
e) &1.64493306684877&9.999994563525405E-7\\ 
f) &Infinity&-Infinity\\ 
g) &1.6449330699290232&9.969192031888952E-7\\ 
h) &1.64493306684877&9.999994563525405E-7\\ 
\end{tabular} 
 \caption{n = 1000000} 
 \end{table}



\begin{table}[ht] 
 \begin{tabular}{l | l | l} 
 
Variante & Endwert & Abweichung vom Grenzwert \\ 
 \hline 

a) &1.0&0.6449340668482264\\ 
b) &Abgebrochen&Abgebrochen\\ 
c) &1.644934057834575&9.013651380840315E-9\\ 
d) &Infinity&-Infinity\\ 
e) &1.644934057834575&9.013651380840315E-9\\ 
f) &Infinity&-Infinity\\ 
g) &1.6449340609148324&5.933393998347469E-9\\ 
h) &1.644934057834575&9.013651380840315E-9\\ 
\end{tabular} 
 \caption{n = 100000000} 
 \end{table}

%/////////////////////// END DOKUMENT %/////////////////////////
\end{document}
