\input{../src/header}											% bindet Header ein (WICHTIG)

\newcommand{\dozent}{Prof. Dr. Claudia Müller-Birn, Barry Linnert}					% <-- Names des Dozenten eintragen
\newcommand{\tutor}{Thierry Meurers}						% <-- Name eurer Tutoriun eintragen
\newcommand{\tutoriumNo}{10}				% <-- Nummer im KVV nachschauen
\newcommand{\ubungNo}{09}									% <-- Nummer des Übungszettels
\newcommand{\veranstaltung}{Objektorientierte Programmierung}	% <-- Name der Lehrveranstaltung eintragen
\newcommand{\semester}{SoSe 17}						% <-- z.B. SoSo 17, WiSe 17/18
\newcommand{\studenten}{Stefaan Hessmann, Jaap Pedersen, Mark Niehues}			% <-- Hier eure Namen eintragen

% /////////////////////// BEGIN DOKUMENT /////////////////////////
\begin{document}
\input{../src/titlepage}										% erstellt die Titelseite


\section{Aufgabe 1}
\lstinputlisting[style=java,									% Style
	caption={House Class},		% Beschriftung
	firstnumber={0},										% Start der Nummerierung
	firstline={0}											% 1. Codezeile
]											% letzte Codezeile
{./src/u9/src/House.java}


\lstinputlisting[style=java,									% Style
	caption={The Street Class organizes the houses},		% Beschriftung
	firstnumber={0},										% Start der Nummerierung
	firstline={0}											% 1. Codezeile
]											% letzte Codezeile
{./src/u9/src/Street.java}

\lstinputlisting[style=java,									% Style
	caption={The Main routine that runs an example simulation with output},		% Beschriftung
	firstnumber={0},										% Start der Nummerierung
	firstline={0}											% 1. Codezeile
]											% letzte Codezeile
{./src/u9/src/Main.java}

%/////////////////////// END DOKUMENT %/////////////////////////
\end{document}
