\input{../src/header}											% bindet Header ein (WICHTIG)

\usepackage{svg}

\newcommand{\dozent}{Prof. Dr. Claudia Müller-Birn, Barry Linnert}					% <-- Names des Dozenten eintragen
\newcommand{\tutor}{Thierry Meurers}						% <-- Name eurer Tutoriun eintragen
\newcommand{\tutoriumNo}{10}				% <-- Nummer im KVV nachschauen
\newcommand{\ubungNo}{11}									% <-- Nummer des Übungszettels
\newcommand{\veranstaltung}{Objektorientierte Programmierung}	% <-- Name der Lehrveranstaltung eintragen
\newcommand{\semester}{SoSe 17}						% <-- z.B. SoSo 17, WiSe 17/18
\newcommand{\studenten}{Stefaan Hessmann, Jaap Pedersen, Mark Niehues}			% <-- Hier eure Namen eintragen

% /////////////////////// BEGIN DOKUMENT /////////////////////////
\begin{document}
\input{../src/titlepage}										% erstellt die Titelseite


\section{Aufgabe 1}
\section{Aufgabe 2}

\section{Aufgabe 3}
Die House-Klasse aus Übungsblatt 09 hat bereits ein Attribut $totalRooms$. Der Konstruktor kann so verändert werden, dass die Anzahl der Zimmer gesetzt werden kann. Dadurch lassen sich sowohl einfache Häuser, als auch Hochhäuser in der Straße verwalten.
\section{Aufgabe 4}
Es kann eine City-Klasse implementiert werden, in der die verschiedenen Straßen verwaltet werden können. Das funktioniert analog zu der Street-Klasse, die eine beliebige Zahl an Häusern verwalten kann.
%/////////////////////// END DOKUMENT %/////////////////////////
\end{document}
