\input{../src/header}											% bindet Header ein (WICHTIG)

\newcommand{\dozent}{Prof. Dr. Claudia Müller-Birn, Barry Linnert}					% <-- Names des Dozenten eintragen
\newcommand{\tutor}{Thierry Meurers}						% <-- Name eurer Tutoriun eintragen
\newcommand{\tutoriumNo}{10}				% <-- Nummer im KVV nachschauen
\newcommand{\ubungNo}{06}									% <-- Nummer des Übungszettels
\newcommand{\veranstaltung}{Objektorientierte Programmierung}	% <-- Name der Lehrveranstaltung eintragen
\newcommand{\semester}{SoSe 17}						% <-- z.B. SoSo 17, WiSe 17/18
\newcommand{\studenten}{Stefaan Hessmann, Jaap Pedersen, Mark Niehues}			% <-- Hier eure Namen eintragen

% /////////////////////// BEGIN DOKUMENT /////////////////////////
\begin{document}
\input{../src/titlepage}										% erstellt die Titelseite


\section{Aufgabe 1}
\begin{description}
\item[a)] 
$2^{\left\lfloor\log_2 n\right\rfloor} \le 2^{\log_2 n} = n$
$\implies 2^{\left\lfloor\log_2 n\right\rfloor} \in \mathcal{O} (n) $
\item[b)] 
$3^{\left\lfloor\log_2 n\right\rfloor}= (2^{\log_2 3})^{\left\lfloor\log_2 n\right\rfloor} \le (2^{\log_2 3})^{\log_2 n} = n^{\log_2 3} \le n^2 $
$\implies 3^{\left\lfloor\log_2 n\right\rfloor} \in \mathcal{O} (n^2) $
\item[c)]
\begin{eqnarray*}
2^{2^{\lfloor \log_2 n \cdot \log_2 n \rfloor}}  &\le & 2^{2^{\log_2 n \cdot \log_2 n + 1}} \\
	& = & 2^{2^{\log_2 n \cdot \log_2 n} \cdot 2} \\
	& = & 2^{n^{\log_2 n} \cdot 2} \\
	& = & 4^{n^{\log_2 n}} \\
\end{eqnarray*}
Mit $ 4^{n^{\log_2 n}} \in \Theta (4^{n^{\log_n n}}) = \Theta (4^{n}) \implies 2^{2^{\lfloor \log_2 n \cdot \log_2 n \rfloor}} \in  \mathcal{O} (4^n)$


\end{description}

\section{Aufgabe 2}
\begin{description}
\item[a)]
Intellij, wegen Bedienung aus pycharm bekannt. 
\item[b)]
Hello World Programm:
\lstinputlisting[style=java,									% Style
	caption={Hello World},		% Beschriftung
	firstnumber={0},										% Start der Nummerierung
	firstline={0}											% 1. Codezeile
]											% letzte Codezeile
{./src/HelloWorld.java}

\end{description}

\section{Aufgabe 3}
\begin{description}
\item[a)] x = 0: 	 Integer
\item[b)] x = false: 	 Boolean
\item[c)] x = true 	: 	olean
\item[d)] x = 7: 	 Integer
\item[e)] x = 1.0: 	 Double
\item[f)] x = 3: 	 Integer
\item[g)] x = false: 	 Boolean
\item[h)] x = 2: 	 Integer
\item[i)] x = -6: 	 Integer
\item[j)] x = 8: 	 Integer
\item[k)] x = 24: 	 Integer
\item[l)] x = Infinity: 	 Double
\end{description}

\section{Aufgabe 4}
\begin{lstlisting}
int a = 2, b = 1, c = 0;
c = a-- + b++; // c = 3, a = 1, b = 2
c = c++; // c = 4
c = --a + b++; // c = 2, a = 0, b = 3
\end{lstlisting}
Nach Ausführung also: $a = 0, b = 3, c = 2$.

\section{Aufgabe 5}
\begin{description}
\item[a)] 
\item[b)] 
\item[c)]
\end{description}



%/////////////////////// END DOKUMENT %/////////////////////////
\end{document}
