\input{../src/header}											% bindet Header ein (WICHTIG)

\newcommand{\dozent}{Prof. Dr. Claudia Müller-Birn, Barry Linnert}					% <-- Names des Dozenten eintragen
\newcommand{\tutor}{Thierry Meurers}						% <-- Name eurer Tutoriun eintragen
\newcommand{\tutoriumNo}{10}				% <-- Nummer im KVV nachschauen
\newcommand{\ubungNo}{02}									% <-- Nummer des Übungszettels
\newcommand{\veranstaltung}{Objektorientierte Programmierung}	% <-- Name der Lehrveranstaltung eintragen
\newcommand{\semester}{SoSe 17}						% <-- z.B. SoSo 17, WiSe 17/18
\newcommand{\studenten}{Stefaan Hessmann, Jaap Pedersen, Mark Niehues}			% <-- Hier eure Namen eintragen

% /////////////////////// BEGIN DOKUMENT /////////////////////////
\begin{document}
\input{../src/titlepage}										% erstellt die Titelseite


% /////////////////////// Aufgabe 1 /////////////////////////
\section{Datentypen in Python}
$<input>: <value>, <type>$
\begin{enumerate}
\item $complex(0): 0+0j, complex$
\item $complex(3): 3+0j, complex$
\item $(1+2j)*(3+0j): 3+6j, complex$
\item $(2+3j)/5j: 0.6-0.4j, complex$
\item $(): (), tuple$
\item $(10): (10), tuple$
\item $[]: [], list$
\item $(0,3)+(1,0): (0,3,1,0), tuple$
\item $2*[0,1]*2: [0,1,0,1,0,1], list$
\item $[1,2,3]+[5,4]: [1,2,3,5,4], list$
\item $\text{2 in (1,3,3)}: False, bool$
\item $2/3: 0.666.., float$
\item $3^16: 19, int$
\item $5|6: 7, int$
\item $9\%7: 2, int$
\item $-3: -3, int$
\item $2<<4: 32, int$
\item $2>>2: 0, int$
\item $-2<<4: -32, int$
\item $-2>>2: -1, int$
\item $1//4+3//4: 0, int$
\item $3**3: 27, int$
\item $0.3+0.1-0.3: 0.1, float$
\item $0.1-0.3: -0.2, float$
\end{enumerate}
% ///////////////////// Aufgabe 2 ///////////////////

\section{Anwendung von Datentypen in Python}
\lstinputlisting[style=py,									% Style
	caption={Output des Programms},		% Beschriftung
	firstnumber={1},										% Start der Nummerierung
	firstline={0}											% 1. Codezeile
]											% letzte Codezeile
{./src/aufgabe_2.py}

\section{Dynamische Typsysteme}
Bei der dynamischen Typisierung wird einer Variablen erst zur Laufzeit ein Datentyp zugeordnet und nicht explizit vorher im Programmcode angegeben. Das ermögicht ggf. mehr Flexibilität und das Programm bleibt in manchen fällen lauffähig obwohl ein nicht vorgeseher Datentyp verwendet wurde (und der Code ist meist besser lesbar). Andererseits fallen auch Zuweisungsfehler erst zur Laufzeit auf.

Quelle: \url{https://de.wikipedia.org/wiki/Typisierung_(Informatik)}

% /////////////////////// END DOKUMENT /////////////////////////
\end{document}
