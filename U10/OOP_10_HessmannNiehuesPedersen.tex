\input{../src/header}											% bindet Header ein (WICHTIG)

\usepackage{svg}

\newcommand{\dozent}{Prof. Dr. Claudia Müller-Birn, Barry Linnert}					% <-- Names des Dozenten eintragen
\newcommand{\tutor}{Thierry Meurers}						% <-- Name eurer Tutoriun eintragen
\newcommand{\tutoriumNo}{10}				% <-- Nummer im KVV nachschauen
\newcommand{\ubungNo}{10}									% <-- Nummer des Übungszettels
\newcommand{\veranstaltung}{Objektorientierte Programmierung}	% <-- Name der Lehrveranstaltung eintragen
\newcommand{\semester}{SoSe 17}						% <-- z.B. SoSo 17, WiSe 17/18
\newcommand{\studenten}{Stefaan Hessmann, Jaap Pedersen, Mark Niehues}			% <-- Hier eure Namen eintragen

% /////////////////////// BEGIN DOKUMENT /////////////////////////
\begin{document}
\input{../src/titlepage}										% erstellt die Titelseite


\section{Aufgabe 1}
Auf der nächsten Seite befindet sich ein kleiner (unvolltändiger) Sketch wie das Programm strukturiert sein könnte. Klassen mit ähnlichem Verhalten (eckig, rund) werden zu abstrakten Subclassen der abstrakten oberklasse aller geometrischen Objekte zusammengefasst. Der Figurehandler verwaltet die Objekte. Dort kann ggf. eine GUI implementiert werden. Die Test-Klasse ist wie der Name schon sagt zum Testen da.
\begin{figure}[htbp]
    \centering
    \def\svgwidth{\columnwidth}
    \input{diagram.pdf_tex}
\end{figure}

%/////////////////////// END DOKUMENT %/////////////////////////
\end{document}
